\documentclass[]{article}
\usepackage{lmodern}
\usepackage{amssymb,amsmath}
\usepackage{ifxetex,ifluatex}
\usepackage{fixltx2e} % provides \textsubscript
\ifnum 0\ifxetex 1\fi\ifluatex 1\fi=0 % if pdftex
  \usepackage[T1]{fontenc}
  \usepackage[utf8]{inputenc}
\else % if luatex or xelatex
  \ifxetex
    \usepackage{mathspec}
  \else
    \usepackage{fontspec}
  \fi
  \defaultfontfeatures{Ligatures=TeX,Scale=MatchLowercase}
\fi
% use upquote if available, for straight quotes in verbatim environments
\IfFileExists{upquote.sty}{\usepackage{upquote}}{}
% use microtype if available
\IfFileExists{microtype.sty}{%
\usepackage{microtype}
\UseMicrotypeSet[protrusion]{basicmath} % disable protrusion for tt fonts
}{}
\usepackage[margin=1in]{geometry}
\usepackage{hyperref}
\hypersetup{unicode=true,
            pdftitle={MSMB-Chapter1-GersteinNotes},
            pdfauthor={Aleeza Gerstein},
            pdfborder={0 0 0},
            breaklinks=true}
\urlstyle{same}  % don't use monospace font for urls
\usepackage{color}
\usepackage{fancyvrb}
\newcommand{\VerbBar}{|}
\newcommand{\VERB}{\Verb[commandchars=\\\{\}]}
\DefineVerbatimEnvironment{Highlighting}{Verbatim}{commandchars=\\\{\}}
% Add ',fontsize=\small' for more characters per line
\usepackage{framed}
\definecolor{shadecolor}{RGB}{248,248,248}
\newenvironment{Shaded}{\begin{snugshade}}{\end{snugshade}}
\newcommand{\KeywordTok}[1]{\textcolor[rgb]{0.13,0.29,0.53}{\textbf{#1}}}
\newcommand{\DataTypeTok}[1]{\textcolor[rgb]{0.13,0.29,0.53}{#1}}
\newcommand{\DecValTok}[1]{\textcolor[rgb]{0.00,0.00,0.81}{#1}}
\newcommand{\BaseNTok}[1]{\textcolor[rgb]{0.00,0.00,0.81}{#1}}
\newcommand{\FloatTok}[1]{\textcolor[rgb]{0.00,0.00,0.81}{#1}}
\newcommand{\ConstantTok}[1]{\textcolor[rgb]{0.00,0.00,0.00}{#1}}
\newcommand{\CharTok}[1]{\textcolor[rgb]{0.31,0.60,0.02}{#1}}
\newcommand{\SpecialCharTok}[1]{\textcolor[rgb]{0.00,0.00,0.00}{#1}}
\newcommand{\StringTok}[1]{\textcolor[rgb]{0.31,0.60,0.02}{#1}}
\newcommand{\VerbatimStringTok}[1]{\textcolor[rgb]{0.31,0.60,0.02}{#1}}
\newcommand{\SpecialStringTok}[1]{\textcolor[rgb]{0.31,0.60,0.02}{#1}}
\newcommand{\ImportTok}[1]{#1}
\newcommand{\CommentTok}[1]{\textcolor[rgb]{0.56,0.35,0.01}{\textit{#1}}}
\newcommand{\DocumentationTok}[1]{\textcolor[rgb]{0.56,0.35,0.01}{\textbf{\textit{#1}}}}
\newcommand{\AnnotationTok}[1]{\textcolor[rgb]{0.56,0.35,0.01}{\textbf{\textit{#1}}}}
\newcommand{\CommentVarTok}[1]{\textcolor[rgb]{0.56,0.35,0.01}{\textbf{\textit{#1}}}}
\newcommand{\OtherTok}[1]{\textcolor[rgb]{0.56,0.35,0.01}{#1}}
\newcommand{\FunctionTok}[1]{\textcolor[rgb]{0.00,0.00,0.00}{#1}}
\newcommand{\VariableTok}[1]{\textcolor[rgb]{0.00,0.00,0.00}{#1}}
\newcommand{\ControlFlowTok}[1]{\textcolor[rgb]{0.13,0.29,0.53}{\textbf{#1}}}
\newcommand{\OperatorTok}[1]{\textcolor[rgb]{0.81,0.36,0.00}{\textbf{#1}}}
\newcommand{\BuiltInTok}[1]{#1}
\newcommand{\ExtensionTok}[1]{#1}
\newcommand{\PreprocessorTok}[1]{\textcolor[rgb]{0.56,0.35,0.01}{\textit{#1}}}
\newcommand{\AttributeTok}[1]{\textcolor[rgb]{0.77,0.63,0.00}{#1}}
\newcommand{\RegionMarkerTok}[1]{#1}
\newcommand{\InformationTok}[1]{\textcolor[rgb]{0.56,0.35,0.01}{\textbf{\textit{#1}}}}
\newcommand{\WarningTok}[1]{\textcolor[rgb]{0.56,0.35,0.01}{\textbf{\textit{#1}}}}
\newcommand{\AlertTok}[1]{\textcolor[rgb]{0.94,0.16,0.16}{#1}}
\newcommand{\ErrorTok}[1]{\textcolor[rgb]{0.64,0.00,0.00}{\textbf{#1}}}
\newcommand{\NormalTok}[1]{#1}
\usepackage{graphicx,grffile}
\makeatletter
\def\maxwidth{\ifdim\Gin@nat@width>\linewidth\linewidth\else\Gin@nat@width\fi}
\def\maxheight{\ifdim\Gin@nat@height>\textheight\textheight\else\Gin@nat@height\fi}
\makeatother
% Scale images if necessary, so that they will not overflow the page
% margins by default, and it is still possible to overwrite the defaults
% using explicit options in \includegraphics[width, height, ...]{}
\setkeys{Gin}{width=\maxwidth,height=\maxheight,keepaspectratio}
\IfFileExists{parskip.sty}{%
\usepackage{parskip}
}{% else
\setlength{\parindent}{0pt}
\setlength{\parskip}{6pt plus 2pt minus 1pt}
}
\setlength{\emergencystretch}{3em}  % prevent overfull lines
\providecommand{\tightlist}{%
  \setlength{\itemsep}{0pt}\setlength{\parskip}{0pt}}
\setcounter{secnumdepth}{0}
% Redefines (sub)paragraphs to behave more like sections
\ifx\paragraph\undefined\else
\let\oldparagraph\paragraph
\renewcommand{\paragraph}[1]{\oldparagraph{#1}\mbox{}}
\fi
\ifx\subparagraph\undefined\else
\let\oldsubparagraph\subparagraph
\renewcommand{\subparagraph}[1]{\oldsubparagraph{#1}\mbox{}}
\fi

%%% Use protect on footnotes to avoid problems with footnotes in titles
\let\rmarkdownfootnote\footnote%
\def\footnote{\protect\rmarkdownfootnote}

%%% Change title format to be more compact
\usepackage{titling}

% Create subtitle command for use in maketitle
\providecommand{\subtitle}[1]{
  \posttitle{
    \begin{center}\large#1\end{center}
    }
}

\setlength{\droptitle}{-2em}

  \title{MSMB-Chapter1-GersteinNotes}
    \pretitle{\vspace{\droptitle}\centering\huge}
  \posttitle{\par}
    \author{Aleeza Gerstein}
    \preauthor{\centering\large\emph}
  \postauthor{\par}
      \predate{\centering\large\emph}
  \postdate{\par}
    \date{2019-09-16}

\usepackage{float}
\let\origfigure\figure
\let\endorigfigure\endfigure
\renewenvironment{figure}[1][2] {
    \expandafter\origfigure\expandafter[H]
} {
    \endorigfigure
}
\let\oldrule=\rule
\renewcommand{\rule}[1]{\oldrule{\linewidth}}

\begin{document}
\maketitle

\section{Chapter 1: Generative models for discrete
data}\label{chapter-1-generative-models-for-discrete-data}

\subsection{Example where we know the probability model for the
process}\label{example-where-we-know-the-probability-model-for-the-process}

Mutation in human genome occur at a rate \(5x10^{-4}\) per nucleotide
per replication cycle. After one cycle, the number of mutations in a
genome of size \(10^4\) = 10,000 nucleotides will follow a \emph{Poisson
distribution} with rate 5. Therefore the \textbf{probability model}
predicts 5 mutations over one replication cycle, and the standard error
is \(\sqrt{5}\).

Thus the \emph{rate parameter} \(\lambda\) = 5 and we can generate the
probability of seeing three events as:

\begin{Shaded}
\begin{Highlighting}[]
\KeywordTok{dpois}\NormalTok{(}\DataTypeTok{x =} \DecValTok{3}\NormalTok{, }\DataTypeTok{lambda =} \DecValTok{5}\NormalTok{)}
\end{Highlighting}
\end{Shaded}

\begin{verbatim}
## [1] 0.1403739
\end{verbatim}

Generate an entire distribution of all values from 0:12 as

\begin{Shaded}
\begin{Highlighting}[]
\NormalTok{.oldopt =}\StringTok{ }\KeywordTok{options}\NormalTok{(}\DataTypeTok{digits =} \DecValTok{2}\NormalTok{)}
\KeywordTok{dpois}\NormalTok{(}\DataTypeTok{x =} \DecValTok{0}\OperatorTok{:}\DecValTok{12}\NormalTok{, }\DataTypeTok{lambda =} \DecValTok{5}\NormalTok{)}
\end{Highlighting}
\end{Shaded}

\begin{verbatim}
##  [1] 0.0067 0.0337 0.0842 0.1404 0.1755 0.1755 0.1462 0.1044 0.0653 0.0363
## [11] 0.0181 0.0082 0.0034
\end{verbatim}

\begin{Shaded}
\begin{Highlighting}[]
\KeywordTok{barplot}\NormalTok{(}\KeywordTok{dpois}\NormalTok{(}\DecValTok{0}\OperatorTok{:}\DecValTok{12}\NormalTok{, }\DecValTok{5}\NormalTok{), }\DataTypeTok{names.arg =} \DecValTok{0}\OperatorTok{:}\DecValTok{12}\NormalTok{, }\DataTypeTok{col =} \StringTok{"red"}\NormalTok{)}
\end{Highlighting}
\end{Shaded}

\begin{figure}
\centering
\includegraphics{MSMB-Chapter1_files/figure-latex/Poisson5-1.pdf}
\caption{Probabilities of seeing 0,1,2,\ldots{},12 mutations, as modeled
by the Poisson(5) distribution. The plot shows that we will often see 4
or 5 mutations but rarely as many as 12. The distribution continues to
higher numbers (\(13,...\)), but the probabilities will be successively
smaller, and here we don't visualize them.}
\end{figure}

\begin{Shaded}
\begin{Highlighting}[]
\KeywordTok{options}\NormalTok{(.oldopt)}
\end{Highlighting}
\end{Shaded}

\subsection{1.3 Using discrete probability
models}\label{using-discrete-probability-models}

There are binary events such as mutations (occurs/does not occur) - a
categorical variable with two \textbf{levels} Other events can have many
different levels.

``When we measure a categorical variable on a sample, we often want to
tally the freequencies of the different levels in a vector of counts. R
calls these variables \textbf{factors}''

Capture the different blood genotypes for 19 subjects in a vector:

\begin{Shaded}
\begin{Highlighting}[]
\NormalTok{genotype <-}\StringTok{ }\KeywordTok{c}\NormalTok{(}\StringTok{"AA"}\NormalTok{,}\StringTok{"AO"}\NormalTok{,}\StringTok{"BB"}\NormalTok{,}\StringTok{"AO"}\NormalTok{,}\StringTok{"OO"}\NormalTok{,}\StringTok{"AO"}\NormalTok{,}\StringTok{"AA"}\NormalTok{,}\StringTok{"BO"}\NormalTok{,}\StringTok{"BO"}\NormalTok{,}
             \StringTok{"AO"}\NormalTok{,}\StringTok{"BB"}\NormalTok{,}\StringTok{"AO"}\NormalTok{,}\StringTok{"BO"}\NormalTok{,}\StringTok{"AB"}\NormalTok{,}\StringTok{"OO"}\NormalTok{,}\StringTok{"AB"}\NormalTok{,}\StringTok{"BB"}\NormalTok{,}\StringTok{"AO"}\NormalTok{,}\StringTok{"AO"}\NormalTok{)}
\KeywordTok{table}\NormalTok{(genotype)}
\end{Highlighting}
\end{Shaded}

\begin{verbatim}
## genotype
## AA AB AO BB BO OO 
##  2  2  7  3  3  2
\end{verbatim}

R automatically detects the factor levels.

\begin{Shaded}
\begin{Highlighting}[]
\NormalTok{genotypeF =}\StringTok{ }\KeywordTok{factor}\NormalTok{(genotype)}

\CommentTok{#You an access the levels liek this:}
\KeywordTok{levels}\NormalTok{(genotypeF)}
\end{Highlighting}
\end{Shaded}

\begin{verbatim}
## [1] "AA" "AB" "AO" "BB" "BO" "OO"
\end{verbatim}

\begin{Shaded}
\begin{Highlighting}[]
\KeywordTok{table}\NormalTok{(genotypeF)}
\end{Highlighting}
\end{Shaded}

\begin{verbatim}
## genotypeF
## AA AB AO BB BO OO 
##  2  2  7  3  3  2
\end{verbatim}

\begin{center}\rule{0.5\linewidth}{\linethickness}\end{center}

\textcolor{red}{Question 1.1  What is you want to create a factor that has some levels not yet in your data?}

\begin{Shaded}
\begin{Highlighting}[]
\NormalTok{genotypeF}\OperatorTok{$}\NormalTok{levels <-}\StringTok{ }\KeywordTok{c}\NormalTok{(}\KeywordTok{levels}\NormalTok{(genotypeF), }\StringTok{"ZZ"}\NormalTok{)}
\end{Highlighting}
\end{Shaded}

\begin{center}\rule{0.5\linewidth}{\linethickness}\end{center}

``If the order in which the data are observed doesn't matter we call the
random variable \textbf{exchangeable}. In this case the vector of
frequencies is \textbf{sufficient} to capture all the relevant
information in the data. It is an effective way to summarize/compress
the data''

\subsubsection{1.3.1 Bernoulli trials}\label{bernoulli-trials}

When there are two possible events with (potentially) unequal
probabilities.

15 bernoulli trials with success equal to 0.5:

\begin{Shaded}
\begin{Highlighting}[]
\KeywordTok{rbinom}\NormalTok{(}\DecValTok{15}\NormalTok{, }\DataTypeTok{prob =} \FloatTok{0.5}\NormalTok{, }\DataTypeTok{size =} \DecValTok{1}\NormalTok{)}
\end{Highlighting}
\end{Shaded}

\begin{verbatim}
##  [1] 1 1 0 0 0 1 1 0 1 1 0 1 1 1 0
\end{verbatim}

Called with **paremeters*; first is n, the number of trials, \(prob\) is
the probability of success, \(size = 1\) indicate each individual trial
is one toss.

12 bernoulli trials with unequal probility of success:

\begin{Shaded}
\begin{Highlighting}[]
\KeywordTok{rbinom}\NormalTok{(}\DecValTok{12}\NormalTok{, }\DataTypeTok{prob =} \DecValTok{2}\OperatorTok{/}\DecValTok{3}\NormalTok{, }\DataTypeTok{size =} \DecValTok{1}\NormalTok{)}
\end{Highlighting}
\end{Shaded}

\begin{verbatim}
##  [1] 1 0 1 0 0 1 1 1 1 0 1 0
\end{verbatim}

\subsubsection{1.3.2 Binomial success
counts}\label{binomial-success-counts}

If you only care about how many successes are in one category then the
order of the trials doesn't matter and you can set the \(size\)
parameter to the number desired and take the sum of the cells in the
output vector

\begin{Shaded}
\begin{Highlighting}[]
\KeywordTok{rbinom}\NormalTok{(}\DecValTok{1}\NormalTok{, }\DataTypeTok{prob =} \DecValTok{2}\OperatorTok{/}\DecValTok{3}\NormalTok{, }\DataTypeTok{size =} \DecValTok{12}\NormalTok{)}
\end{Highlighting}
\end{Shaded}

\begin{verbatim}
## [1] 8
\end{verbatim}

\begin{center}\rule{0.5\linewidth}{\linethickness}\end{center}

\textcolor{red}{Question 1.2: Repeat this function call a number of times. Why isn't the answer always the same?}
Beause it is a probability function.

\begin{center}\rule{0.5\linewidth}{\linethickness}\end{center}

This is also called a random two-box model, when there are only two
possible outcomes; we only need to specify \emph{p} the probability of
success since the \emph{complementary} event will occur with probability
1 - \emph{p}. If the events are independent of each other
(\emph{exchangeable}) we only record the number of successes.

The number of successes in 15 bernoulli trials with a probabilty of
success 0.3 is called a **binomial* random variable that follows the
B(15, 0.3) distribution. To generate samples:

\begin{Shaded}
\begin{Highlighting}[]
\KeywordTok{set.seed}\NormalTok{(}\DecValTok{235569515}\NormalTok{)}
\KeywordTok{rbinom}\NormalTok{(}\DecValTok{1}\NormalTok{, }\DataTypeTok{prob =} \FloatTok{0.3}\NormalTok{, }\DataTypeTok{size =} \DecValTok{15}\NormalTok{)}
\end{Highlighting}
\end{Shaded}

\begin{verbatim}
## [1] 5
\end{verbatim}

\begin{center}\rule{0.5\linewidth}{\linethickness}\end{center}

\textcolor{red}{Question 1.3: Repeat this function call 10 times. What seems to be the most common outcome?}

\begin{Shaded}
\begin{Highlighting}[]
\NormalTok{i <-}\StringTok{ }\DecValTok{0}
\NormalTok{trial <-}\StringTok{ }\KeywordTok{c}\NormalTok{()}
\ControlFlowTok{while}\NormalTok{(i }\OperatorTok{<}\StringTok{ }\DecValTok{11}\NormalTok{)\{}
\NormalTok{  i <-}\StringTok{ }\NormalTok{i }\OperatorTok{+}\DecValTok{1}
\NormalTok{  trial[i] <-}\StringTok{ }\KeywordTok{rbinom}\NormalTok{(}\DecValTok{1}\NormalTok{, }\DataTypeTok{prob =} \FloatTok{0.3}\NormalTok{, }\DataTypeTok{size =} \DecValTok{15}\NormalTok{)}
\NormalTok{\}}
\NormalTok{  trial}
\end{Highlighting}
\end{Shaded}

\begin{verbatim}
##  [1] 5 2 2 4 5 3 5 7 2 5 2
\end{verbatim}

or

\begin{Shaded}
\begin{Highlighting}[]
\NormalTok{.freqoutcome <-}\StringTok{ }\NormalTok{(}\DecValTok{0}\OperatorTok{:}\DecValTok{15}\NormalTok{)[}\KeywordTok{which.max}\NormalTok{(}\KeywordTok{dbinom}\NormalTok{(}\DecValTok{0}\OperatorTok{:}\DecValTok{15}\NormalTok{, }\DataTypeTok{prob =} \FloatTok{0.3}\NormalTok{, }\DecValTok{15}\NormalTok{))]}
\NormalTok{.freqoutcome}
\end{Highlighting}
\end{Shaded}

\begin{verbatim}
## [1] 4
\end{verbatim}

\begin{center}\rule{0.5\linewidth}{\linethickness}\end{center}

The complete \textbf{probability mass distribution} is available by
typing:

\begin{Shaded}
\begin{Highlighting}[]
\NormalTok{probabilities =}\StringTok{ }\KeywordTok{dbinom}\NormalTok{(}\DecValTok{0}\OperatorTok{:}\DecValTok{15}\NormalTok{, }\DataTypeTok{prob =} \FloatTok{0.3}\NormalTok{, }\DataTypeTok{size =} \DecValTok{15}\NormalTok{)}
\KeywordTok{round}\NormalTok{(probabilities, }\DecValTok{2}\NormalTok{)}
\end{Highlighting}
\end{Shaded}

\begin{verbatim}
##  [1] 0.00 0.03 0.09 0.17 0.22 0.21 0.15 0.08 0.03 0.01 0.00 0.00 0.00 0.00
## [15] 0.00 0.00
\end{verbatim}

\begin{Shaded}
\begin{Highlighting}[]
\KeywordTok{c}\NormalTok{(}\DecValTok{0}\OperatorTok{:}\DecValTok{15}\NormalTok{)[}\KeywordTok{which}\NormalTok{(probabilities}\OperatorTok{==}\KeywordTok{max}\NormalTok{(probabilities))]}
\end{Highlighting}
\end{Shaded}

\begin{verbatim}
## [1] 4
\end{verbatim}

Barplot of above probabilities

\begin{Shaded}
\begin{Highlighting}[]
\KeywordTok{barplot}\NormalTok{(probabilities, }\DataTypeTok{names.arg =} \DecValTok{0}\OperatorTok{:}\DecValTok{15}\NormalTok{, }\DataTypeTok{col =} \StringTok{"red"}\NormalTok{)}
\end{Highlighting}
\end{Shaded}

\begin{figure}
\centering
\includegraphics{MSMB-Chapter1_files/figure-latex/binombarplot-1.pdf}
\caption{Theoretical distribution of \(B(15,0.3)\) . The highest bar is
at \(x = 4\). We have chosen to represent theoretical values in red
throughout.}
\end{figure}

The number of trial is the \emph{size} parameter and is often written
\emph{n} while the probability of success i \emph{p}. For \(X\)
distributed as a binomial distribution with parameters (\emph{n},
\emph{p}), written \(X\) \textasciitilde{} \(B(n,p)\), the probability
of seeing \(X\) = \(k\) success is

P(\(X\) = k) = \((\frac{n}{k})p^k(1-p)^{n-k}\)

\begin{center}\rule{0.5\linewidth}{\linethickness}\end{center}

\textcolor{red}{Question 1.4: What is the output of the formula for k = 3, p = 2/3, n = 4?}

\begin{Shaded}
\begin{Highlighting}[]
\KeywordTok{dbinom}\NormalTok{(}\DecValTok{3}\NormalTok{, }\DataTypeTok{prob =} \DecValTok{2}\OperatorTok{/}\DecValTok{3}\NormalTok{, }\DataTypeTok{size =} \DecValTok{4}\NormalTok{)}
\end{Highlighting}
\end{Shaded}

\begin{verbatim}
## [1] 0.3950617
\end{verbatim}

\begin{center}\rule{0.5\linewidth}{\linethickness}\end{center}

\subsection{1.3.3 Poisson distributions}\label{poisson-distributions}

When the probability of success \emph{p} is small and the number of
trial \emph{n} is large, the binomial distribution \(B(n, p)\) can be
faithfully approximated by a simpler distribution, the \textbf{poisson
distribution} with rate parameter \(\lambda = np\).

\begin{center}\rule{0.5\linewidth}{\linethickness}\end{center}

\textcolor{red}{Question 1.5: what is the probability mass distribution of observing 0:12 mutations ina  genome of $n = 10^4$ nucleotides, when the probability of mutation is *p* = 5 $\times$ $10^{-4}$ per nucleotide? Is it similar when modeled by the binomial $B(n, p)$ distribution and by the poisson $\lambda = np$ distribution?}

\includegraphics{MSMB-Chapter1_files/figure-latex/testbinomvspois-1.pdf}

\begin{center}\rule{0.5\linewidth}{\linethickness}\end{center}

The Poisson depends only on the product of \(np\). The mathematical
formula is: \(P(X = k) = \frac{\lambda^ke^{-k}}{k!}\)

For instance, take \(\lambda\) = 5 and compute \(P(X = 3)\)

\begin{Shaded}
\begin{Highlighting}[]
\DecValTok{5}\OperatorTok{^}\DecValTok{3} \OperatorTok{*}\StringTok{ }\KeywordTok{exp}\NormalTok{(}\OperatorTok{-}\DecValTok{5}\NormalTok{) }\OperatorTok{/}\StringTok{ }\KeywordTok{factorial}\NormalTok{(}\DecValTok{3}\NormalTok{)}
\end{Highlighting}
\end{Shaded}

\begin{verbatim}
## [1] 0.1403739
\end{verbatim}

\textbackslash{}textcolor\{red\}\{Task: simulate a mutation process
along 10, 000 positions with a mutation rate of \(5\times10^{-4}\) and
count the number of mutations. Repat this many times and plot the
distribution.

\begin{Shaded}
\begin{Highlighting}[]
\KeywordTok{rbinom}\NormalTok{(}\DecValTok{1}\NormalTok{, }\DataTypeTok{prob =} \FloatTok{5e-4}\NormalTok{, }\DataTypeTok{size =} \DecValTok{10000}\NormalTok{)}
\end{Highlighting}
\end{Shaded}

\begin{verbatim}
## [1] 3
\end{verbatim}

\begin{Shaded}
\begin{Highlighting}[]
\NormalTok{simulations =}\StringTok{ }\KeywordTok{rbinom}\NormalTok{(}\DataTypeTok{n =} \DecValTok{300000}\NormalTok{, }\DataTypeTok{prob =} \FloatTok{5e-4}\NormalTok{, }\DataTypeTok{size =} \DecValTok{10000}\NormalTok{)}
\KeywordTok{barplot}\NormalTok{(}\KeywordTok{table}\NormalTok{(simulations), }\DataTypeTok{col =} \StringTok{"lavender"}\NormalTok{)}
\end{Highlighting}
\end{Shaded}

\begin{figure}
\centering
\includegraphics{MSMB-Chapter1_files/figure-latex/gen-simpoisson-1.pdf}
\caption{Simulated distribution of B(10000, \(10^{-4}\)) for
(ref:gen-simpoisson-1) simulations.}
\end{figure}

\subsubsection{1.3.4 A generative model for epitope
detection}\label{a-generative-model-for-epitope-detection}

ELISA assay the detects epitopes at 100 different independent positions.
The false positive rate is 1\% We can say that p(declare epitope
\textbar{} no epitope) = 1\% or, declaring an epitope when there is no
epitope is 1\%. The general form is *\(X|Y\) as ``given'' or
``conditional on''. Thus, ``X happens conditional on Y being the
case''.``* . The data for one patien looks like this:

\textcolor{blue}{WHAT IS THIS NOMENCLATURE?}

\begin{center}\rule{0.5\linewidth}{\linethickness}\end{center}

\textcolor{red}{Task: Verify by simulation that the sum of 50 independent Bernoulli variables with p = 0.01 is - to good enough approximation - the same as a Poisson(0.5) random variable}

If there are no epitopes and the counts follow a Poisson(0.01)
distribution. Each individual position and patient has a probability of
1 in 100 of being a 1. At any given position after seeing 50 patients,
we expect the sum to have a Poisson distribution with parameter 0.5.

\begin{Shaded}
\begin{Highlighting}[]
\NormalTok{bernoul <-}\StringTok{ }\KeywordTok{c}\NormalTok{()}
\NormalTok{poiss <-}\StringTok{ }\KeywordTok{c}\NormalTok{()}
\ControlFlowTok{for}\NormalTok{(i }\ControlFlowTok{in} \DecValTok{1}\OperatorTok{:}\DecValTok{100}\NormalTok{)\{}
\NormalTok{  bernoul[i] <-}\StringTok{ }\KeywordTok{sum}\NormalTok{(}\KeywordTok{rbinom}\NormalTok{(}\DataTypeTok{n =} \DecValTok{100}\NormalTok{, }\DataTypeTok{prob =} \FloatTok{0.01}\NormalTok{, }\DataTypeTok{size =} \DecValTok{50}\NormalTok{))}
\NormalTok{  poiss[i] <-}\StringTok{ }\KeywordTok{sum}\NormalTok{(}\KeywordTok{rpois}\NormalTok{(}\DecValTok{100}\NormalTok{, }\DataTypeTok{lambda =} \FloatTok{0.5}\NormalTok{))}
\NormalTok{\}}
\KeywordTok{hist}\NormalTok{(bernoul, }\DataTypeTok{xlim=}\KeywordTok{c}\NormalTok{(}\DecValTok{0}\NormalTok{, }\DecValTok{100}\NormalTok{), }\DataTypeTok{breaks=}\DecValTok{50}\NormalTok{, }\DataTypeTok{col=}\StringTok{"blue"}\NormalTok{, }\DataTypeTok{main=}\StringTok{""}\NormalTok{, }\DataTypeTok{ylim=}\KeywordTok{c}\NormalTok{(}\DecValTok{0}\NormalTok{, }\DecValTok{15}\NormalTok{), }\DataTypeTok{xlab =}\StringTok{"sum of detections after 50 patients"}\NormalTok{, }\DataTypeTok{ylab =} \StringTok{"frequency"}\NormalTok{)}
\KeywordTok{legend}\NormalTok{(}\StringTok{"topright"}\NormalTok{, }\DataTypeTok{col=}\KeywordTok{c}\NormalTok{(}\StringTok{"blue"}\NormalTok{, }\StringTok{"red"}\NormalTok{), }\DataTypeTok{legend=}\KeywordTok{c}\NormalTok{(}\StringTok{"bernoulli"}\NormalTok{, }\StringTok{"poisson"}\NormalTok{), }\DataTypeTok{pch=}\DecValTok{22}\NormalTok{, }\DataTypeTok{pt.bg =}\KeywordTok{c}\NormalTok{(}\StringTok{"blue"}\NormalTok{, }\StringTok{"red"}\NormalTok{))}
\KeywordTok{par}\NormalTok{(}\DataTypeTok{new=}\NormalTok{T)}
\KeywordTok{hist}\NormalTok{(poiss, }\DataTypeTok{breaks=}\DecValTok{40}\NormalTok{, }\DataTypeTok{col=}\StringTok{"red"}\NormalTok{, }\DataTypeTok{xlim=}\KeywordTok{c}\NormalTok{(}\DecValTok{0}\NormalTok{, }\DecValTok{100}\NormalTok{), }\DataTypeTok{main=}\StringTok{""}\NormalTok{, }\DataTypeTok{ylim=}\KeywordTok{c}\NormalTok{(}\DecValTok{0}\NormalTok{, }\DecValTok{15}\NormalTok{), }\DataTypeTok{xlab=}\StringTok{""}\NormalTok{, }\DataTypeTok{ylab=}\StringTok{""}\NormalTok{)}
\end{Highlighting}
\end{Shaded}

\includegraphics{MSMB-Chapter1_files/figure-latex/unnamed-chunk-4-1.pdf}

\begin{figure}
\centering
\includegraphics{MSMB-Chapter1_files/figure-latex/typicalP-1.pdf}
\caption{Plot of typical data from our generative model for the
background, i.,e., for the false positive hits: 100 positions along the
protein, at each position the count is drawn from a Poisson(0.5) random
variable.}
\end{figure}

\begin{figure}
\centering
\includegraphics{MSMB-Chapter1_files/figure-latex/epitopedata-1.pdf}
\caption{Output of the ELISA array results for 50 patients in the 100
positions.}
\end{figure}

What are the chances of seeing a value as large as 7, if no epitope is
present?

If you look for the probability of seeing a number as big as 7 (or
larger) when considering one Poisson(0.5) random variable, you can
calculate the closed form as \(\sum_{k=7}^{\infty} P(X=k)\)

This is the same as 1 - P(X \(<=\) 6).

P(X \(<=\) 6) is the \textbf{cumulative distribution function} at 6 and
we use the function \texttt{ppois} for computing it.

\begin{Shaded}
\begin{Highlighting}[]
\DecValTok{1} \OperatorTok{-}\StringTok{ }\KeywordTok{ppois}\NormalTok{(}\DecValTok{6}\NormalTok{, }\FloatTok{0.5}\NormalTok{)}
\end{Highlighting}
\end{Shaded}

\begin{verbatim}
## [1] 1.00238e-06
\end{verbatim}

\begin{Shaded}
\begin{Highlighting}[]
\KeywordTok{ppois}\NormalTok{(}\DecValTok{6}\NormalTok{, }\FloatTok{0.5}\NormalTok{, }\DataTypeTok{lower.tail =} \OtherTok{FALSE}\NormalTok{)}
\end{Highlighting}
\end{Shaded}

\begin{verbatim}
## [1] 1.00238e-06
\end{verbatim}

\begin{center}\rule{0.5\linewidth}{\linethickness}\end{center}

\textcolor{blue}{Task: What is the meaning of `lower.tail`?} ``Setting
lower.tail = FALSE allows to get much more precise results when the
default, lower.tail = TRUE would return 1, see the example below.''

\begin{Shaded}
\begin{Highlighting}[]
\DecValTok{1} \OperatorTok{-}\StringTok{ }\KeywordTok{ppois}\NormalTok{(}\DecValTok{10}\OperatorTok{*}\NormalTok{(}\DecValTok{15}\OperatorTok{:}\DecValTok{25}\NormalTok{), }\DataTypeTok{lambda =} \DecValTok{100}\NormalTok{)  }\CommentTok{# becomes 0 (cancellation)}
\end{Highlighting}
\end{Shaded}

\begin{verbatim}
##  [1] 1.233094e-06 1.261664e-08 7.085799e-11 2.252643e-13 4.440892e-16
##  [6] 0.000000e+00 0.000000e+00 0.000000e+00 0.000000e+00 0.000000e+00
## [11] 0.000000e+00
\end{verbatim}

\begin{Shaded}
\begin{Highlighting}[]
\KeywordTok{ppois}\NormalTok{(}\DecValTok{10}\OperatorTok{*}\NormalTok{(}\DecValTok{15}\OperatorTok{:}\DecValTok{25}\NormalTok{), }\DataTypeTok{lambda =} \DecValTok{100}\NormalTok{, }\DataTypeTok{lower.tail =} \OtherTok{FALSE}\NormalTok{)  }\CommentTok{# no cancellation}
\end{Highlighting}
\end{Shaded}

\begin{verbatim}
##  [1] 1.233094e-06 1.261664e-08 7.085800e-11 2.253110e-13 4.174239e-16
##  [6] 4.626179e-19 3.142097e-22 1.337219e-25 3.639328e-29 6.453883e-33
## [11] 7.587807e-37
\end{verbatim}

? ***

(1.1) We use
\(\epsilon = P(X \geq 7) = 1 - P(X \leq 6) \approx 10^{-6}\) (1.1)

** Extreme value analysis for the poisson distirubiton **

\textcolor{red}{Question 1.6: The above calculation is not correct if we want to compute the probability that we observe these data if there is no epitope}

We looked at all positions to find the maximum (7), which is more likely
to occur than if we looked at only one position. So instead of asking
what are the chances of seeing a Poisson(0.5) as large as 7, we should
ask what are the chances that the maximum of 100 Poisson(0.5) trials is
as large as 7. Use \textbf{extreme value theorem} Order the data
\(x_1, x_2, .. x_{100}\) so that \(x_{1}\) is the smallest value and
\(x_{100}\) is the largest of the counts over the 100 positions.
Together this set \(x_1, x_2, .. x_{100}\) is called the \textbf{rank
statistic} of this sample of 100 values.

Then the maximum value being as large as 7 is the \emph{complementary
event} of having all 100 counts be smaller or equal to 6. Two
\emph{complementary events} have probabilities that sum to 1.

\[P(x_{(100)}\geq 7)=1-P(x_{(100)} \leq 6)=1-
\Pi_{i=1}^{100} P(x_i \leq 6 )\]

Because we suppose these 100 events are independent, we can use our
result from 1.1 above:

\textcolor{blue}{huh.} \[
\Pi_{i=1}^{100} P(x_i \leq 6 ) = (P(x_i \leq 6))^{100} = (1- \epsilon)^{100}
\]

\textbf{Actually calculating the numbers}

\textbf{Computing probabilities by simulation}

Use a \emph{monte carlo} method: a computer simulation based on our
generative model that finds the probabilities of the events we're
interested in. Essentially generate Figure 1.7 again and again and see
how often the biggest spike is 7 or greater.

\begin{Shaded}
\begin{Highlighting}[]
\NormalTok{maxes =}\StringTok{ }\KeywordTok{replicate}\NormalTok{(}\DecValTok{100000}\NormalTok{, \{}
  \KeywordTok{max}\NormalTok{(}\KeywordTok{rpois}\NormalTok{(}\DecValTok{100}\NormalTok{, }\FloatTok{0.5}\NormalTok{))}
\NormalTok{\})}
\KeywordTok{table}\NormalTok{(maxes)}
\end{Highlighting}
\end{Shaded}

\begin{verbatim}
## maxes
##     1     2     3     4     5     6     7     9 
##     8 23072 60805 14354  1604   141    15     1
\end{verbatim}

\begin{Shaded}
\begin{Highlighting}[]
\KeywordTok{mean}\NormalTok{( maxes }\OperatorTok{>=}\StringTok{ }\DecValTok{7}\NormalTok{ )}
\end{Highlighting}
\end{Shaded}

\begin{verbatim}
## [1] 0.00016
\end{verbatim}

That's the same as the number of times maxes was 7 or greater divided by
100000. Potential limitation of Monte Carlo simulations is that the
inverse of the number of simulations is the `granularity'. What we have
done is an example of \textbf{probability or generative modeling}: all
parameters are known and the theory allows us to work by
\textbf{deduction} in a top-down fashion.

If we knew instead the number of patients and the length of the proteins
but did not know the distribution of the data then we have to use
\textbf{statistical modeling}. If we only have the data then we
\textbf{fit} a reasonable distirubiton to describe it.

\subsection{1.4 Multinomial distributions: the case of
DNA}\label{multinomial-distributions-the-case-of-dna}

\textbf{More than two outcomes}

\begin{center}\rule{0.5\linewidth}{\linethickness}\end{center}

\textbackslash{}textcolor\{blue\}\{Task: Experiment with the random
number generator that generates all possible numbers between 0 and 1
through the function called \texttt{runif}. Use it to generate random
variable with four levels (A, C, T, G) where \(p_A = \frac{1}{8}\),
\(p_B = \frac{3}{8}\), \(p_C = \frac{3}{8}\), \(p_G = \frac{1}{8}\)

\begin{Shaded}
\begin{Highlighting}[]
\NormalTok{?sample}
\end{Highlighting}
\end{Shaded}

\begin{center}\rule{0.5\linewidth}{\linethickness}\end{center}

\textbf{Mathematical formulation}

\textcolor{red}{Question 1.7: Suppose we have four boxes that are equally likely. Using the formula, what is the probability of observing four in the first box, two in the second box and none in the two other boxes?}

\begin{Shaded}
\begin{Highlighting}[]
\KeywordTok{dmultinom}\NormalTok{(}\KeywordTok{c}\NormalTok{(}\DecValTok{4}\NormalTok{, }\DecValTok{2}\NormalTok{, }\DecValTok{0}\NormalTok{, }\DecValTok{0}\NormalTok{), }\DataTypeTok{prob =} \KeywordTok{rep}\NormalTok{(}\DecValTok{1}\OperatorTok{/}\DecValTok{4}\NormalTok{, }\DecValTok{4}\NormalTok{))}
\end{Highlighting}
\end{Shaded}

\begin{verbatim}
## [1] 0.003662109
\end{verbatim}

\textcolor{blue}{Formula?}

Run simulations to test whether the data are consistent with each box
having the same probability. Suppose eight characters of four differentm
equally likely types:

\begin{Shaded}
\begin{Highlighting}[]
\NormalTok{pvec =}\StringTok{ }\KeywordTok{rep}\NormalTok{(}\DecValTok{1}\OperatorTok{/}\DecValTok{4}\NormalTok{, }\DecValTok{4}\NormalTok{)}
\KeywordTok{t}\NormalTok{(}\KeywordTok{rmultinom}\NormalTok{(}\DecValTok{1}\NormalTok{, }\DataTypeTok{prob =}\NormalTok{ pvec, }\DataTypeTok{size =} \DecValTok{8}\NormalTok{))}
\end{Highlighting}
\end{Shaded}

\begin{verbatim}
##      [,1] [,2] [,3] [,4]
## [1,]    3    2    2    1
\end{verbatim}

\begin{center}\rule{0.5\linewidth}{\linethickness}\end{center}

\texcolor{red}{Question 1.9: How do you interpret the difference between `rmultinom(n=9, prob=pvec, size =1)` and `rmultimon(n=1, prob = pvec, size =8)`?}

n is the number of random draws size is the number of objects that are
put into K boxes

\begin{Shaded}
\begin{Highlighting}[]
\KeywordTok{t}\NormalTok{(}\KeywordTok{rmultinom}\NormalTok{(}\DataTypeTok{n=}\DecValTok{8}\NormalTok{, }\DataTypeTok{prob=}\NormalTok{pvec, }\DataTypeTok{size =}\DecValTok{1}\NormalTok{)) }\CommentTok{#8 trials with one ball thrown into 1 of 4 boxes}
\end{Highlighting}
\end{Shaded}

\begin{verbatim}
##      [,1] [,2] [,3] [,4]
## [1,]    0    1    0    0
## [2,]    0    0    1    0
## [3,]    0    0    1    0
## [4,]    0    1    0    0
## [5,]    1    0    0    0
## [6,]    0    0    0    1
## [7,]    0    1    0    0
## [8,]    1    0    0    0
\end{verbatim}

\begin{Shaded}
\begin{Highlighting}[]
\KeywordTok{t}\NormalTok{(}\KeywordTok{rmultinom}\NormalTok{(}\DataTypeTok{n=}\DecValTok{1}\NormalTok{, }\DataTypeTok{prob =}\NormalTok{ pvec, }\DataTypeTok{size =}\DecValTok{8}\NormalTok{)) }\CommentTok{#one trial with 8 balls thrown into 4 boxes}
\end{Highlighting}
\end{Shaded}

\begin{verbatim}
##      [,1] [,2] [,3] [,4]
## [1,]    1    1    4    2
\end{verbatim}

\begin{center}\rule{0.5\linewidth}{\linethickness}\end{center}

\subsection{1.4.1 Simulating for Power}\label{simulating-for-power}

Important Q: How big a sample size do I need?

\emph{Power} is the probability of detecting something if it is there,
i.e., the \emph{true positive rate} Conventionally want power of
\(\geq\) 80\%. i.e., if the same experiment is run many times, about
20\% of the time it will fail to yield significant results even it it
should.

Generate 1000 simulations from the null hypothesis using the
\texttt{rmultinom} function. Display only the first 11 columns.

\begin{Shaded}
\begin{Highlighting}[]
\NormalTok{obsunder0 =}\StringTok{ }\KeywordTok{rmultinom}\NormalTok{(}\DecValTok{1000}\NormalTok{, }\DataTypeTok{prob =}\NormalTok{ pvec, }\DataTypeTok{size =} \DecValTok{20}\NormalTok{)}
\KeywordTok{dim}\NormalTok{(obsunder0)}
\end{Highlighting}
\end{Shaded}

\begin{verbatim}
## [1]    4 1000
\end{verbatim}

\begin{Shaded}
\begin{Highlighting}[]
\NormalTok{obsunder0[, }\DecValTok{1}\OperatorTok{:}\DecValTok{11}\NormalTok{]}
\end{Highlighting}
\end{Shaded}

\begin{verbatim}
##      [,1] [,2] [,3] [,4] [,5] [,6] [,7] [,8] [,9] [,10] [,11]
## [1,]    6    3    5    7    8    6    5    4    9     6     2
## [2,]    7    9    3    3    5    3    3    6    3     4     4
## [3,]    2    6    8    4    4    5    7    7    7     7     5
## [4,]    5    2    4    6    3    6    5    3    1     3     9
\end{verbatim}

Eah column is a simulated instance. Teh numbers vary a lot. The expected
value is 20/4 = 5 but in some cases it goes up much higher.

\subsubsection{Creating a test}\label{creating-a-test}

Knowing the expected value (1/4) isn't enough. We also need a measure of
\textbf{variability} to describe how much variability is expected and
how much is too much. Use the following statistic that is computed as
the sum of squares of the squre of the differences between the observed
calue and the expected value relative to the expected value. I.e.,
\(\chi\)-square.

stat = \(\sum_i = \frac{(E_i - x_i)^2}{E_i}\)

How do the first three columns of the generated data differ from what we
expect?

\begin{Shaded}
\begin{Highlighting}[]
\NormalTok{expected0 =}\StringTok{ }\NormalTok{pvec }\OperatorTok{*}\StringTok{ }\DecValTok{20}
\KeywordTok{sum}\NormalTok{((obsunder0[, }\DecValTok{1}\NormalTok{] }\OperatorTok{-}\StringTok{ }\NormalTok{expected0)}\OperatorTok{^}\DecValTok{2} \OperatorTok{/}\StringTok{ }\NormalTok{expected0)}
\end{Highlighting}
\end{Shaded}

\begin{verbatim}
## [1] 2.8
\end{verbatim}

\begin{Shaded}
\begin{Highlighting}[]
\KeywordTok{sum}\NormalTok{((obsunder0[, }\DecValTok{2}\NormalTok{] }\OperatorTok{-}\StringTok{ }\NormalTok{expected0)}\OperatorTok{^}\DecValTok{2} \OperatorTok{/}\StringTok{ }\NormalTok{expected0)}
\end{Highlighting}
\end{Shaded}

\begin{verbatim}
## [1] 6
\end{verbatim}

\begin{Shaded}
\begin{Highlighting}[]
\KeywordTok{sum}\NormalTok{((obsunder0[, }\DecValTok{3}\NormalTok{] }\OperatorTok{-}\StringTok{ }\NormalTok{expected0)}\OperatorTok{^}\DecValTok{2} \OperatorTok{/}\StringTok{ }\NormalTok{expected0)}
\end{Highlighting}
\end{Shaded}

\begin{verbatim}
## [1] 2.8
\end{verbatim}

The values differ. Encapsulate the formula for \texttt{stat} in a
fumction.

\begin{Shaded}
\begin{Highlighting}[]
\NormalTok{stat =}\StringTok{ }\ControlFlowTok{function}\NormalTok{(obsvd, }\DataTypeTok{exptd =} \DecValTok{20} \OperatorTok{*}\StringTok{ }\NormalTok{pvec) \{}
  \KeywordTok{sum}\NormalTok{((obsvd }\OperatorTok{-}\StringTok{ }\NormalTok{exptd)}\OperatorTok{^}\DecValTok{2} \OperatorTok{/}\StringTok{ }\NormalTok{exptd)}
\NormalTok{\}}
\KeywordTok{stat}\NormalTok{(obsunder0[, }\DecValTok{1}\NormalTok{])}
\end{Highlighting}
\end{Shaded}

\begin{verbatim}
## [1] 2.8
\end{verbatim}

Then compute the measure for all 1000 instances and call it \(S0\) which
contains values generated under \(H_0\). COnsider the histogram of
\(S0\) to be the \emph{null distribution}.

\begin{Shaded}
\begin{Highlighting}[]
\NormalTok{S0 =}\StringTok{ }\KeywordTok{apply}\NormalTok{(obsunder0, }\DecValTok{2}\NormalTok{, stat)}
\KeywordTok{summary}\NormalTok{(S0)}
\end{Highlighting}
\end{Shaded}

\begin{verbatim}
##    Min. 1st Qu.  Median    Mean 3rd Qu.    Max. 
##   0.000   1.200   2.400   2.888   4.000  14.000
\end{verbatim}

\begin{Shaded}
\begin{Highlighting}[]
\KeywordTok{hist}\NormalTok{(S0, }\DataTypeTok{breaks =} \DecValTok{50}\NormalTok{, }\DataTypeTok{col =} \StringTok{"lavender"}\NormalTok{, }\DataTypeTok{main =} \StringTok{""}\NormalTok{)}
\end{Highlighting}
\end{Shaded}

\begin{figure}
\centering
\includegraphics{MSMB-Chapter1_files/figure-latex/histS0-1.pdf}
\caption{The histogram of simulated values \texttt{S0} of the statistic
\texttt{stat} under the null (fair) distribution provides an
approximation of the \textbf{sampling distribution} of the statistic
\texttt{stat}.}
\end{figure}

From this can approximate the 95\% quantile:

\begin{Shaded}
\begin{Highlighting}[]
\NormalTok{q95 =}\StringTok{ }\KeywordTok{quantile}\NormalTok{(S0, }\DataTypeTok{probs =} \FloatTok{0.95}\NormalTok{)}
\NormalTok{q95}
\end{Highlighting}
\end{Shaded}

\begin{verbatim}
##  95% 
## 7.22
\end{verbatim}

That tells us that 5\% of the values are larger than 7.6. This becomes
our criteria for testing: we wil reject the null hypothesis that the
data come from a fair proess if the weighted sum of squares `stat' is
larger than 7.6.

\emph{Determine the test power} Now compute the probability that our
test will detect data that do not come from the null hypothesis.

\begin{Shaded}
\begin{Highlighting}[]
\NormalTok{pvecA =}\StringTok{ }\KeywordTok{c}\NormalTok{(}\DecValTok{3}\OperatorTok{/}\DecValTok{8}\NormalTok{, }\DecValTok{1}\OperatorTok{/}\DecValTok{4}\NormalTok{, }\DecValTok{3}\OperatorTok{/}\DecValTok{12}\NormalTok{, }\DecValTok{1}\OperatorTok{/}\DecValTok{8}\NormalTok{)}
\NormalTok{observed =}\StringTok{ }\KeywordTok{rmultinom}\NormalTok{(}\DecValTok{1000}\NormalTok{, }\DataTypeTok{prob =}\NormalTok{ pvecA, }\DataTypeTok{size =} \DecValTok{20}\NormalTok{)}
\KeywordTok{dim}\NormalTok{(observed)}
\end{Highlighting}
\end{Shaded}

\begin{verbatim}
## [1]    4 1000
\end{verbatim}

\begin{Shaded}
\begin{Highlighting}[]
\NormalTok{observed[, }\DecValTok{1}\OperatorTok{:}\DecValTok{7}\NormalTok{]}
\end{Highlighting}
\end{Shaded}

\begin{verbatim}
##      [,1] [,2] [,3] [,4] [,5] [,6] [,7]
## [1,]    7    4    7    8    4    8    8
## [2,]    8    8    4    6    5    3    5
## [3,]    3    4    6    3    7    8    5
## [4,]    2    4    3    3    4    1    2
\end{verbatim}

\begin{Shaded}
\begin{Highlighting}[]
\KeywordTok{apply}\NormalTok{(observed, }\DecValTok{1}\NormalTok{, mean)}
\end{Highlighting}
\end{Shaded}

\begin{verbatim}
## [1] 7.407 4.973 5.088 2.532
\end{verbatim}

\begin{Shaded}
\begin{Highlighting}[]
\NormalTok{expectedA =}\StringTok{ }\NormalTok{pvecA }\OperatorTok{*}\StringTok{ }\DecValTok{20}
\NormalTok{expectedA}
\end{Highlighting}
\end{Shaded}

\begin{verbatim}
## [1] 7.5 5.0 5.0 2.5
\end{verbatim}

\begin{Shaded}
\begin{Highlighting}[]
\KeywordTok{stat}\NormalTok{(observed[, }\DecValTok{1}\NormalTok{])}
\end{Highlighting}
\end{Shaded}

\begin{verbatim}
## [1] 5.2
\end{verbatim}

\begin{Shaded}
\begin{Highlighting}[]
\NormalTok{S1 =}\StringTok{ }\KeywordTok{apply}\NormalTok{(observed, }\DecValTok{2}\NormalTok{, stat)}
\NormalTok{q95}
\end{Highlighting}
\end{Shaded}

\begin{verbatim}
##  95% 
## 7.22
\end{verbatim}

\begin{Shaded}
\begin{Highlighting}[]
\KeywordTok{sum}\NormalTok{(S1 }\OperatorTok{>}\StringTok{ }\NormalTok{q95)}
\end{Highlighting}
\end{Shaded}

\begin{verbatim}
## [1] 235
\end{verbatim}

\begin{Shaded}
\begin{Highlighting}[]
\NormalTok{power =}\StringTok{ }\KeywordTok{mean}\NormalTok{(S1 }\OperatorTok{>}\StringTok{ }\NormalTok{q95)}
\NormalTok{power}
\end{Highlighting}
\end{Shaded}

\begin{verbatim}
## [1] 0.235
\end{verbatim}

Thus with a sequence length of 20 we have a power of about 20\% to
detect difference between a fair generating process and our
\textbf{alternative}.

\begin{center}\rule{0.5\linewidth}{\linethickness}\end{center}

\textcolor{red}{Suggest a new sequence length that will ensure the power is acceptable}

\begin{Shaded}
\begin{Highlighting}[]
\NormalTok{expectedA =}\StringTok{ }\NormalTok{pvecA }\OperatorTok{*}\StringTok{ }\DecValTok{20}
\ControlFlowTok{for}\NormalTok{(i }\ControlFlowTok{in} \DecValTok{20}\OperatorTok{:}\DecValTok{30}\NormalTok{)\{}
\NormalTok{  observed =}\StringTok{ }\KeywordTok{rmultinom}\NormalTok{(}\DecValTok{1000}\NormalTok{, }\DataTypeTok{prob =}\NormalTok{ pvecA, }\DataTypeTok{size =}\NormalTok{ i)}
\NormalTok{  S1 =}\StringTok{ }\KeywordTok{apply}\NormalTok{(observed, }\DecValTok{2}\NormalTok{, stat)}
\NormalTok{  power =}\StringTok{ }\KeywordTok{mean}\NormalTok{(S1 }\OperatorTok{>}\StringTok{ }\NormalTok{q95)}
  \KeywordTok{print}\NormalTok{(}\KeywordTok{paste0}\NormalTok{(i ,}\StringTok{" = "}\NormalTok{, power))}
\NormalTok{\}}
\end{Highlighting}
\end{Shaded}

\begin{verbatim}
## [1] "20 = 0.271"
## [1] "21 = 0.285"
## [1] "22 = 0.315"
## [1] "23 = 0.398"
## [1] "24 = 0.48"
## [1] "25 = 0.547"
## [1] "26 = 0.605"
## [1] "27 = 0.737"
## [1] "28 = 0.798"
## [1] "29 = 0.856"
## [1] "30 = 0.923"
\end{verbatim}

\texttt{\{r\ ProbaDiagram,\ eval\ =\ TRUE,\ echo\ =\ FALSE,\ fig.show\ =\ \textquotesingle{}hold\textquotesingle{},\ fig.keep\ =\ \textquotesingle{}high\textquotesingle{},\ fig.cap\ =\ "We\ have\ studied\ how\ a\ probability\ model\ has\ a\ distribution\ we\ call\ this\ \$F\$.\ \$F\$\ often\ depends\ on\ parameters,\ which\ are\ denoted\ by\ Greek\ letters,\ such\ as\ \$\textbackslash{}\textbackslash{}theta\$.\ The\ observed\ data\ are\ generated\ via\ the\ brown\ arrow\ and\ are\ represented\ by\ Roman\ letters\ such\ as\ \$x\$.\ The\ vertical\ bar\ in\ the\ probability\ computation\ stands\ for\ **supposing\ that**\ or\ **conditional\ on**"-\/-\/-\/-\ knitr::include\_graphics(c(\textquotesingle{}images/ProbaDiagram.png\textquotesingle{}))}

\begin{Shaded}
\begin{Highlighting}[]
\KeywordTok{dbinom}\NormalTok{(}\DecValTok{2}\NormalTok{, }\DataTypeTok{size =} \DecValTok{10}\NormalTok{, }\DataTypeTok{prob =} \FloatTok{0.3}\NormalTok{)}
\end{Highlighting}
\end{Shaded}

\begin{verbatim}
## [1] 0.2334744
\end{verbatim}

\begin{Shaded}
\begin{Highlighting}[]
\KeywordTok{pbinom}\NormalTok{(}\DecValTok{2}\NormalTok{, }\DataTypeTok{size =} \DecValTok{10}\NormalTok{, }\DataTypeTok{prob =} \FloatTok{0.3}\NormalTok{)}
\end{Highlighting}
\end{Shaded}

\begin{verbatim}
## [1] 0.3827828
\end{verbatim}

\begin{Shaded}
\begin{Highlighting}[]
\KeywordTok{sum}\NormalTok{(}\KeywordTok{sapply}\NormalTok{(}\DecValTok{0}\OperatorTok{:}\DecValTok{2}\NormalTok{, dbinom, }\DataTypeTok{size =} \DecValTok{10}\NormalTok{, }\DataTypeTok{prob =} \FloatTok{0.3}\NormalTok{))}
\end{Highlighting}
\end{Shaded}

\begin{verbatim}
## [1] 0.3827828
\end{verbatim}

\begin{Shaded}
\begin{Highlighting}[]
\KeywordTok{poismax}\NormalTok{(}\DataTypeTok{lambda =} \FloatTok{0.5}\NormalTok{, }\DataTypeTok{n =} \DecValTok{100}\NormalTok{, }\DataTypeTok{m =} \DecValTok{7}\NormalTok{)}
\end{Highlighting}
\end{Shaded}

\begin{verbatim}
## [1] 0.0001002329
\end{verbatim}

\begin{Shaded}
\begin{Highlighting}[]
\KeywordTok{poismax}\NormalTok{(}\DataTypeTok{lambda =} \KeywordTok{mean}\NormalTok{(e100), }\DataTypeTok{n =} \DecValTok{100}\NormalTok{, }\DataTypeTok{m =} \DecValTok{7}\NormalTok{)}
\end{Highlighting}
\end{Shaded}

\begin{verbatim}
## [1] 0.0001870183
\end{verbatim}

\begin{Shaded}
\begin{Highlighting}[]
\ControlFlowTok{if}\NormalTok{ (}\OperatorTok{!}\KeywordTok{requireNamespace}\NormalTok{(}\StringTok{"BiocManager"}\NormalTok{, }\DataTypeTok{quietly =} \OtherTok{TRUE}\NormalTok{))}
     \KeywordTok{install.packages}\NormalTok{(}\StringTok{"BiocManager"}\NormalTok{)}
\NormalTok{BiocManager}\OperatorTok{::}\KeywordTok{install}\NormalTok{(}\KeywordTok{c}\NormalTok{(}\StringTok{"Biostrings"}\NormalTok{, }\StringTok{"BSgenome.Celegans.UCSC.ce2"}\NormalTok{))}
\end{Highlighting}
\end{Shaded}

\begin{Shaded}
\begin{Highlighting}[]
\KeywordTok{library}\NormalTok{(}\StringTok{"BSgenome.Celegans.UCSC.ce2"}\NormalTok{)}
\NormalTok{Celegans}
\KeywordTok{seqnames}\NormalTok{(Celegans)}
\NormalTok{Celegans}\OperatorTok{$}\NormalTok{chrM}
\KeywordTok{class}\NormalTok{(Celegans}\OperatorTok{$}\NormalTok{chrM)}
\KeywordTok{length}\NormalTok{(Celegans}\OperatorTok{$}\NormalTok{chrM)}
\end{Highlighting}
\end{Shaded}

\begin{Shaded}
\begin{Highlighting}[]
\KeywordTok{library}\NormalTok{(}\StringTok{"Biostrings"}\NormalTok{)}
\NormalTok{lfM =}\StringTok{ }\KeywordTok{letterFrequency}\NormalTok{(Celegans}\OperatorTok{$}\NormalTok{chrM, }\DataTypeTok{letters=}\KeywordTok{c}\NormalTok{(}\StringTok{"A"}\NormalTok{, }\StringTok{"C"}\NormalTok{, }\StringTok{"G"}\NormalTok{, }\StringTok{"T"}\NormalTok{))}
\NormalTok{lfM}
\KeywordTok{sum}\NormalTok{(lfM)}
\NormalTok{lfM }\OperatorTok{/}\StringTok{ }\KeywordTok{sum}\NormalTok{(lfM)}
\end{Highlighting}
\end{Shaded}

\begin{Shaded}
\begin{Highlighting}[]
\KeywordTok{t}\NormalTok{(}\KeywordTok{rmultinom}\NormalTok{(}\DecValTok{1}\NormalTok{, }\KeywordTok{length}\NormalTok{(Celegans}\OperatorTok{$}\NormalTok{chrM), }\DataTypeTok{p =} \KeywordTok{rep}\NormalTok{(}\DecValTok{1}\OperatorTok{/}\DecValTok{4}\NormalTok{, }\DecValTok{4}\NormalTok{)))}
\end{Highlighting}
\end{Shaded}

\begin{Shaded}
\begin{Highlighting}[]
\KeywordTok{length}\NormalTok{(Celegans}\OperatorTok{$}\NormalTok{chrM) }\OperatorTok{/}\StringTok{ }\DecValTok{4}
\end{Highlighting}
\end{Shaded}

\begin{Shaded}
\begin{Highlighting}[]
\NormalTok{oestat =}\StringTok{ }\ControlFlowTok{function}\NormalTok{(o, e) \{}
  \KeywordTok{sum}\NormalTok{((e}\OperatorTok{-}\NormalTok{o)}\OperatorTok{^}\DecValTok{2} \OperatorTok{/}\StringTok{ }\NormalTok{e)}
\NormalTok{\}}
\NormalTok{oe =}\StringTok{ }\KeywordTok{oestat}\NormalTok{(}\DataTypeTok{o =}\NormalTok{ lfM, }\DataTypeTok{e =} \KeywordTok{length}\NormalTok{(Celegans}\OperatorTok{$}\NormalTok{chrM) }\OperatorTok{/}\StringTok{ }\DecValTok{4}\NormalTok{)}
\NormalTok{oe}
\end{Highlighting}
\end{Shaded}

\begin{Shaded}
\begin{Highlighting}[]
\NormalTok{B =}\StringTok{ }\DecValTok{10000}
\NormalTok{n =}\StringTok{ }\KeywordTok{length}\NormalTok{(Celegans}\OperatorTok{$}\NormalTok{chrM)}
\NormalTok{expected =}\StringTok{ }\KeywordTok{rep}\NormalTok{(n }\OperatorTok{/}\StringTok{ }\DecValTok{4}\NormalTok{, }\DecValTok{4}\NormalTok{)}
\NormalTok{oenull =}\StringTok{ }\KeywordTok{replicate}\NormalTok{(B,}
  \KeywordTok{oestat}\NormalTok{(}\DataTypeTok{e =}\NormalTok{ expected, }\DataTypeTok{o =} \KeywordTok{rmultinom}\NormalTok{(}\DecValTok{1}\NormalTok{, n, }\DataTypeTok{p =} \KeywordTok{rep}\NormalTok{(}\DecValTok{1}\OperatorTok{/}\DecValTok{4}\NormalTok{, }\DecValTok{4}\NormalTok{))))}
\end{Highlighting}
\end{Shaded}

\subsubsection{Question 1.1}\label{question-1.1}

\begin{Shaded}
\begin{Highlighting}[]
\NormalTok{genotype <-}\StringTok{ }\KeywordTok{c}\NormalTok{(}\StringTok{"AA"}\NormalTok{,}\StringTok{"AO"}\NormalTok{,}\StringTok{"BB"}\NormalTok{,}\StringTok{"AO"}\NormalTok{,}\StringTok{"OO"}\NormalTok{,}\StringTok{"AO"}\NormalTok{,}\StringTok{"AA"}\NormalTok{,}\StringTok{"BO"}\NormalTok{,}\StringTok{"BO"}\NormalTok{,}
             \StringTok{"AO"}\NormalTok{,}\StringTok{"BB"}\NormalTok{,}\StringTok{"AO"}\NormalTok{,}\StringTok{"BO"}\NormalTok{,}\StringTok{"AB"}\NormalTok{,}\StringTok{"OO"}\NormalTok{,}\StringTok{"AB"}\NormalTok{,}\StringTok{"BB"}\NormalTok{,}\StringTok{"AO"}\NormalTok{,}\StringTok{"AO"}\NormalTok{)}
\CommentTok{#How to create a factor that has some levcels not yet in your data?}
\NormalTok{genotypeF =}\StringTok{ }\KeywordTok{factor}\NormalTok{(genotype, }\DataTypeTok{levels =} \KeywordTok{c}\NormalTok{(}\KeywordTok{unique}\NormalTok{(genotype), }\StringTok{"CC"}\NormalTok{))}
\KeywordTok{levels}\NormalTok{(genotypeF)}
\end{Highlighting}
\end{Shaded}

\begin{verbatim}
## [1] "AA" "AO" "BB" "OO" "BO" "AB" "CC"
\end{verbatim}

\begin{Shaded}
\begin{Highlighting}[]
\KeywordTok{table}\NormalTok{(genotypeF)}
\end{Highlighting}
\end{Shaded}

\begin{verbatim}
## genotypeF
## AA AO BB OO BO AB CC 
##  2  7  3  2  3  2  0
\end{verbatim}

\subsubsection{Question 1.4}\label{question-1.4}

\begin{Shaded}
\begin{Highlighting}[]
\KeywordTok{dbinom}\NormalTok{(}\DecValTok{3}\NormalTok{, }\DataTypeTok{prob =} \DecValTok{2}\OperatorTok{/}\DecValTok{3}\NormalTok{, }\DataTypeTok{size =} \DecValTok{4}\NormalTok{)}
\end{Highlighting}
\end{Shaded}

\begin{verbatim}
## [1] 0.3950617
\end{verbatim}

\subsubsection{Question 1.5}\label{question-1.5}

\begin{Shaded}
\begin{Highlighting}[]
\NormalTok{probabilities_Q5_binom <-}\StringTok{ }\KeywordTok{dbinom}\NormalTok{(}\DecValTok{0}\OperatorTok{:}\DecValTok{12}\NormalTok{, }\DataTypeTok{prob =} \FloatTok{5E-4}\NormalTok{, }\DataTypeTok{size =} \FloatTok{1E4}\NormalTok{)}
\NormalTok{probabilities_Q5_poisson <-}\StringTok{ }\KeywordTok{dpois}\NormalTok{(}\DecValTok{0}\OperatorTok{:}\DecValTok{12}\NormalTok{, }\DataTypeTok{lambda =} \FloatTok{5E-4}\OperatorTok{*}\FloatTok{1E4}\NormalTok{)}
\end{Highlighting}
\end{Shaded}


\end{document}
